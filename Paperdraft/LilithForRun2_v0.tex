% =========================================================================
% SciPost LaTeX template
% Version 1e (2017-10-31)
% =========================================================================

% For submitting a paper to SciPost Physics: 
\documentclass[submission, Phys]{SciPost}

\linenumbers

\begin{document}

% title
\begin{center}{\Large \textbf{
  Constraining new physics from Higgs measurements with\\[1mm] Lilith: update to LHC Run~2 results}}\end{center}

% Authors; mark the corresponding author with a superscript *.
\begin{center}
Thi Nhung Dao\textsuperscript{1},
Sabine Kraml\textsuperscript{2*},
Duc Ninh Le\textsuperscript{1},
Loc Tran Quang\textsuperscript{1}
\end{center}

% Affiliations
\begin{center}
{\bf 1} Institute For Interdisciplinary Research in Science and Education, ICISE,\\ 590000, Quy Nhon, Vietnam\\
{\bf 2} Laboratoire de Physique Subatomique et de Cosmologie, Universit\'e Grenoble-Alpes,\\ CNRS/IN2P3, 53 Avenue des Martyrs, F-38026 Grenoble, France\\
% email address of corresponding author
* sabine.kraml@lpsc.in2p3.fr
\end{center}

\begin{center}
\today
\end{center}

% For convenience during refereeing: line numbers
%\linenumbers

\section*{Abstract}
{\bf
Lilith is public python library for constraining new physics from Higgs signal strength measurements. 
We here present an update of Lilith (version 1.2) including ATLAS and CMS Run~2 Higgs results for 36~fb$^{-1}$.  
Both the code and the XML database where extended from the ordinary Gaussian approximation employed in 
Lilith-1.1 to using variable Gaussian and Poisson distributions.  
We provide detailed validations of the implemented experimental results as well as 
a status of global fits for {\it i)} reduced Higgs couplings and {\it ii)} Two-Higgs-doublet models of Type-I and Type-II. 
We also comment on shortcomings in the presentation of the experimental results, which make their re-use difficult. 
Lilith-1.2 is available on GitHub and ready to be used to constrain a wide class of new physics scenarios.}


% include a table of contents if paper is longer than 6 pages
%\vspace{10pt}
%\noindent\rule{\textwidth}{1pt}
%\tableofcontents\thispagestyle{fancy}
%\noindent\rule{\textwidth}{1pt}
%\vspace{10pt}


%===================================================================================
\section{Introduction} \label{sec:intro}
%===================================================================================

...........\\
...........\\
...........\\
...........\\
...........\\
...........\\


\clearpage
%===================================================================================
\section{Extended XML format} \label{sec:intro}
%===================================================================================

In the Lilith database, every single experimental result is stored in a different XML file. 
The original input formats from~\cite{Bernon:2015hsa} are 
\begin{itemize} 
\item 1D intervals: best fit with $1\sigma$ errors; 
\item 2D likelihood contours: best fit, confidence level and parameters a, b, c which parametrize the inverse of the covariance matrix;
\item full likelihood information as 1D or 2D grids of $-2\log L$.
\end{itemize}
For a detailed discussion and description of the XML format, we refer the reader to the original Lilith manual~\cite{Bernon:2015hsa}. 
Here we just note that the first two options, 1D intervals and 2D likelihood contours, rely on an ordinary Gaussian approximation, 
which does not always describe the experimental data (i.e.\ the true likelihood) very well. 
Full 2D likelihood grids would be ideal but are rarely available.\footnote{We note that \cite{Boudjema:2013qla} strongly advocated 
the publication of full likelihood grids in 2 or more dimensions but unfortunately this wasn't followed up by the experimental collaborations.} 
We have therefore extended the XML format and fitting procedure in Lilith to include 
\begin{itemize} 
\item Gaussian distributions of variable width (``variable Gaussian'') and 
\item generalized Poisson distributions. 
\end{itemize}
The format follows the structure defined in \cite{Bernon:2015hsa} but setting {\tt type="vn"} for a variable Gaussian and
{\tt type="p"} for a Poisson distribution in the {\tt <expmu>} tag. % (with {\tt dim="1"} for 1D and {\tt dim="2"} for 2D as before). 
Concretely:

\paragraph{1D variable Gaussian} 

An example is the ATLAS result for $\mu(VH, ZZ)$ from HIGG-2016-22 which gives a 95\% CL limit of $\mu<3.7$. 
We assume that the best fit lies at zero, leading to a one-sided $1\sigma$ uncertainty of $1.85$. 

\begin{verbatim}
<expmu decay="ZZ" dim="1" type="vn">
  <experiment>ATLAS</experiment>
  <source type="published">HIGG-2016-22</source>
  <sqrts>13</sqrts>
  <mass>125.09</mass>

  <eff prod="VH">1.0</eff>

  <bestfit>0.</bestfit>
  
  <param>
  <uncertainty side="left">-0.0</uncertainty>
  <uncertainty side="right">1.85</uncertainty>
  </param>
</expmu>
\end{verbatim}

\noindent
The {\tt <bestfit>} tag contains the best-fit value for the signal strength. 
The {\tt <uncertainty>} tag contains the left (negative) and right (positive) $1\sigma$ errors.
The computation of the likelihood in {\tt computelikelihood.py} then follows Section~3.6 ``Variable Gaussian (2)'' of \cite{Barlow:2004wg}. 


\paragraph{2D variable Gaussian} 

Taking the ATLAS result for $\mu(VBF, ZZ)$ and $\mu(ggH, ZZ)$ with correlation $\rho=-0.41$ from HIGG-2016-22 as an example:

\begin{verbatim}
<expmu decay="ZZ" dim="2" type="vn">
  <experiment>ATLAS</experiment>
  <source type="published">HIGG-2016-22</source>
  <sqrts>13</sqrts>
  <mass>125.09</mass>
  <CL>68\%</CL>  <!-- optional -->

  <eff axis="x" prod="VBF">1.0</eff>
  <eff axis="y" prod="ggH">1.0</eff>

  <bestfit>
    <x>4.0</x>
    <y>1.11</y>
  </bestfit>
 
  <param>
    <uncertainty axis="x" side="left">-1.46</uncertainty>
    <uncertainty axis="x" side="right">+1.75</uncertainty>
    <uncertainty axis="y" side="left">-0.21</uncertainty>
    <uncertainty axis="y" side="right">+0.23</uncertainty>
    <correlation>-0.41</correlation>
  </param>
</expmu>
\end{verbatim}

\noindent
Here, the {\tt <bestfit>} tag specifies the location of the best-fit point in the ({\tt x,y}) plane. 
The {\tt <uncertainty>} tags contain the left (negative) and right (positive) $1\sigma$ errors for the {\tt x} and {\tt y} axes. 
The {\tt <correlation>} tag specifies the correlation between {\tt x} and {\tt y}. 
The computation of the likelihood again follows Section~3.6 ``Variable Gaussian (2)'' of \cite{Barlow:2004wg}. 

\paragraph{1D Poisson} 

\begin{verbatim}
<expmu decay="gammagamma" dim="1" type="p">
  <experiment>CMS</experiment>
  <source type="published">HIG-16-040</source>
  <sqrts>13</sqrts>
  <mass>125</mass>

  <eff prod="VH">1.</eff>

  <bestfit>2.4</bestfit>

  <param>
    <alpha>6.66515303803</alpha>
    <nu>48.9221714101</nu>
  </param>
</expmu>
\end{verbatim}

\noindent 
The {\tt <bestfit>} tag again contains the best-fit value for the signal strength, 
while the tags {\tt <alpha>} and {\tt <nu>} specify the scaling and skew of the function 
according to Section~3.4 ``Generalised Poisson'', eq.~(10a), of \cite{Barlow:2004wg}. 


\paragraph{2D Poisson}   
 
 {\em ....... TODO: add explanation ...........}\\
...........\\
...........\\
...........\\


%\begin{verbatim}
%<expmu decay="gammagamma" dim="2" type="p">
%...
%...
%\end{verbatim}


%\clearpage
%===================================================================================
\section{ATLAS and CMS results included in the database update}
%===================================================================================

%-----------------------------------------------------------------------------------------------
\subsection{ATLAS Run~2 results for 36~fb$^{-1}$}
%-----------------------------------------------------------------------------------------------

\subsubsection*{\boldmath $H\to\gamma\gamma$: HIGG-2016-21}

The ATLAS analysis \cite{Aaboud:2018xdt} provides in Fig.~12 the 1D signal strengths measured for the different production processes:  
ggH, VBF, VH and ``top'' (ttH+tH). Moreover, the paper provides a variety of results for simplified template cross sections (STXS) 
including, in Fig.~40a, the observed correlations between the measured stage-0 STXS. 
Likelihood contours at 68\% and 95\% CL in the $\sigma(ggH)\times {\rm BR}(H\to\gamma\gamma)$ vs.\ 
$\sigma(VBF)\times {\rm BR}(H\to\gamma\gamma)$ plane are shown in Fig.~15 of \cite{Aaboud:2018xdt}. 
Finally, auxiliary Figs.~23a--d show the 1D profile likelihoods of the signal strengths for ggH, VBF, VH and top production modes.
As validation material, we have likelihood contours in the $C_g$ vs.\ $C_\gamma$ plane in Fig.~18a and in the $C_V$ vs.\ $C_F$ plane in Fig.~18b.\footnote{The reduced couplings $\kappa_X$ in the experimental papers are denoted as $C_X$ in Lilith.}
With no HepDATA record available for this analysis, all these figures had to be digitized ``by hand''.

It turns out that the ordinary Gaussian approximation, using the 1D signal strengths with their correlations or 
a bivariate Gaussian distribution fitted from the 68\% CL contour of Fig.~15 (normalized to SM), does not describe the data well.
%does not reproduce well the $C_g$ vs.\ $C_\gamma$ and $C_V$ vs.\ $C_F$ fits of \cite{Aaboud:2018xdt} Fig.~18. This is illustrated in the left-hand panels in Figure~\ref{validation_atlas_gamgam}. 
In fact, the 1D profile likelihoods of the signal strengths have a Poisson shape. 
We therefore parametrize $\mu(ggH,\gamma\gamma)$ vs.\ $\mu(VBF,\gamma\gamma)$ as a 2D Poisson distribution with correlation $-0.27$,  starting from the 1D profile likelihoods from auxiliary Figs.~23a,b of \cite{Aaboud:2018xdt}. For $\mu(VH,\gamma\gamma)$ and $\mu(ttH,\gamma\gamma)$, we use 1D Poisson distributions fitted from auxiliary Figs.~23c,d. 
The impact on the $C_V$ vs.\ $C_F$ fits from using Gaussian or Poisson likelihoods is illustrated 
in Figure~\ref{validation_atlas_gamgam}. Clearly, the Poissonian case reproduces much better the official ATLAS fits. 

\begin{figure}[h!]%\centering
\hspace*{-8mm}\includegraphics[width=0.43\textwidth]{validation/ATLAS/HIGG-2016-21-CVCF-Gaussian.pdf}%
\hspace*{-12mm}\includegraphics[width=0.43\textwidth]{validation/ATLAS/HIGG-2016-21-CVCF-GaussianV2.pdf}%
\hspace*{-12mm}\includegraphics[width=0.43\textwidth]{validation/ATLAS/HIGG-2016-21-CVCF-Poisson.pdf} 
\caption{Fits of $C_V$ vs.\ $C_F$ using data from the ATLAS $H\to\gamma\gamma$ measurement~\cite{Aaboud:2018xdt}; 
the signal strength measurements are implemented, from left to right, as ordinary Gaussian, variable Gaussian and Poisson likelihoods.}
\label{validation_atlas_gamgam}
\end{figure}
 
\subsubsection*{\boldmath $H\to ZZ^*\to 4l$: HIGG-2016-22}


\subsubsection*{\boldmath $H\to WW^*\to 2l2\nu$: HIGG-2016-07}

\subsubsection*{\boldmath $H\to \tau\tau$: HIGG-2017-07}

\subsubsection*{\boldmath $H\to b\bar b$: HIGG-2016-29 (VH) and HIGG-2016-30 (VBF)}

\subsubsection*{\boldmath $ttH$}

\subsubsection*{\boldmath $H\to inv$: HIGG-2016-28}

Results from the search for invisibly decaying Higgs bosons produced in association with a $Z$ boson are presented in \cite{Aaboud:2017bja}. Assuming the Standard Model $ZH$ production cross-section, an observed (expected) upper limit of 67\% (39\%) at the 95\% confidence level is set on BR$(H\to inv)$ for $m_H= 125$~GeV. We use $1-{\rm CLs}$ as function BR$(H\to inv)$ extracted from auxiliary Figure~1c on the analysis' webpage. 


%-----------------------------------------------------------------------------------------------
\subsection{CMS Run~2 results for 36~fb$^{-1}$}
%-----------------------------------------------------------------------------------------------


%===================================================================================
\section{Status of Higgs coupling fits}
%===================================================================================


%===================================================================================
\section{Conclusion}
%===================================================================================
 must include a conclusion.

%===================================================================================
\section*{Acknowledgements}
%===================================================================================

S.K.~thanks W.~Adam, R.~Sch\"ofbeck, W.~Waltenberger and N.~Wardle for helpful discussions. 

%\paragraph{Author contributions}
%This is optional. If desired, contributions should be succinctly described in a single short paragraph, using author initials.

%\paragraph{Funding information}
%Authors are required to provide funding information, including relevant agencies and grant numbers with linked author's initials. Correctly-provided data will be linked to funders listed in the \href{https://www.crossref.org/services/funder-registry/}{\sf Fundref registry}.
This work was supported by the IN2P3 theory project 
``LHC-itools: methods and tools for the interpretation of the LHC Run~2 results for new physics''. 
D.T.N.\ thanks the LPSC Grenoble for hospitality and financial support for a research visit within the LHC-itools project. 
L.T.Q.\ thanks the ICISE ...


%===================================================================================
\begin{appendix}
%===================================================================================

\section{Overview of XML data files}

\section{Implementation of 2D Poisson likelihood with correlation}


\end{appendix}


%===================================================================================
% References
%===================================================================================

% \bibliographystyle{SciPost_bibstyle} % Include this style file here only if you are not using our template
\bibliography{references.bib}

\nolinenumbers

\end{document}
