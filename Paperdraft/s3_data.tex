\clearpage
%===================================================================================
\section{ATLAS and CMS results included in the database update}
%===================================================================================


%-----------------------------------------------------------------------------------------------
\subsection{ATLAS Run~2 results for 36~fb$^{-1}$}
%-----------------------------------------------------------------------------------------------

The ATLAS Run~2 results included in this release are summarised in Table~\ref{tab:ATLASresults} and explained in more detail below.

\begin{table}[h]\centering
\begin{tabular}{l | ccccccc}
mode & $\gamma\gamma$ & $ZZ$ & $WW$ & $\tau\tau$ & $b\bar b$ & inv. \\
\hline
ggH & \cite{Aaboud:2018xdt} & \cite{Aaboud:2017vzb} & \cite{Aaboud:2018jqu} & \cite{Aaboud:2018pen} & -- & --\\
VBF &  \cite{Aaboud:2018xdt} & \cite{Aaboud:2017vzb} & \cite{Aaboud:2018jqu} & \cite{Aaboud:2018pen} & \cite{Aaboud:2018gay} & -- \\
WH & \multirow{2}{*}{\!\!\cite{Aaboud:2018xdt}} & \multirow{2}{*}{\!\!\cite{Aaboud:2017vzb}} & -- & -- & \cite{Aaboud:2017xsd} & -- \\
ZH &  &  & -- & -- & \cite{Aaboud:2017xsd} & \cite{Aaboud:2017bja} \\
ttH & \cite{Aaboud:2018xdt,Aaboud:2017jvq} & \cite{Aaboud:2017vzb,Aaboud:2017jvq} & \cite{Aaboud:2017jvq} & \cite{Aaboud:2017jvq} & \cite{Aaboud:2017jvq,Aaboud:2017rss} & -- \\ 
\end{tabular}
\caption{Overview of ATLAS Run~2 results included in this release.} 
\label{tab:ATLASresults}
\end{table}


\subsubsection*{\boldmath $H\to\gamma\gamma$: HIGG-2016-21}

The ATLAS analysis \cite{Aaboud:2018xdt} provides in Fig.~12 the 1D signal strengths measured for the different production processes:  
ggH, VBF, VH and ``top'' (ttH+tH). Moreover, the paper provides a variety of results for simplified template cross sections (STXS) 
including, in Fig.~40a, the observed correlations between the measured stage-0 STXS. 
Likelihood contours at 68\% and 95\% CL in the $\sigma(ggH)\times {\rm BR}(H\to\gamma\gamma)$ vs.\ 
$\sigma(VBF)\times {\rm BR}(H\to\gamma\gamma)$ plane are shown in Fig.~15 of \cite{Aaboud:2018xdt}. 
Finally, auxiliary Figs.~23a--d show the 1D profile likelihoods of the signal strengths for ggH, VBF, VH and top production modes.
As validation material, we have likelihood contours in the $C_g$ vs.\ $C_\gamma$ plane in Fig.~18a and in the $C_V$ vs.\ $C_F$ plane in Fig.~18b.\footnote{The reduced couplings $\kappa_X$ in the experimental papers are denoted as $C_X$ in Lilith.}
With no HepDATA record available for this analysis, all these figures had to be digitized ``by hand''.

It turns out that the ordinary Gaussian approximation, using the 1D signal strengths with their correlations or 
a bivariate Gaussian distribution fitted from the 68\% CL contour of Fig.~15 (normalized to SM), does not describe the data well.
%does not reproduce well the $C_g$ vs.\ $C_\gamma$ and $C_V$ vs.\ $C_F$ fits of \cite{Aaboud:2018xdt} Fig.~18. This is illustrated in the left-hand panels in Figure~\ref{validation_atlas_gamgam}. 
In fact, the 1D profile likelihoods of the signal strengths have a Poisson shape. 
We therefore parametrize $\mu(ggH,\gamma\gamma)$ vs.\ $\mu(VBF,\gamma\gamma)$ as a 2D Poisson distribution with correlation $-0.27$,  starting from the 1D profile likelihoods from auxiliary Figs.~23a,b of \cite{Aaboud:2018xdt}. For $\mu(VH,\gamma\gamma)$ and $\mu(ttH,\gamma\gamma)$, we use 1D Poisson distributions fitted from auxiliary Figs.~23c,d. 
The impact on the $C_V$ vs.\ $C_F$ fits from using Gaussian or Poisson likelihoods is illustrated 
in Figure~\ref{validation_atlas_gamgam}. Clearly, the Poissonian case reproduces much better the official ATLAS fits. 

%\begin{figure}[h!]%\centering
%\hspace*{-8mm}\includegraphics[width=0.43\textwidth]{validation/ATLAS/HIGG-2016-21-CVCF-Gaussian.pdf}%
%\hspace*{-12mm}\includegraphics[width=0.43\textwidth]{validation/ATLAS/HIGG-2016-21-CVCF-GaussianV2.pdf}%
%\hspace*{-12mm}\includegraphics[width=0.43\textwidth]{validation/ATLAS/HIGG-2016-21-CVCF-Poisson.pdf} 
%\caption{Fits of $C_V$ vs.\ $C_F$ using data from the ATLAS $H\to\gamma\gamma$ measurement~\cite{Aaboud:2018xdt}; 
%the signal strength measurements are implemented, from left to right, as ordinary Gaussian, variable Gaussian and Poisson likelihoods.}
%\label{validation_atlas_gamgam}
%\end{figure}
 
\subsubsection*{\boldmath $H\to ZZ^*\to 4l$: HIGG-2016-22}


\subsubsection*{\boldmath $H\to WW^*\to 2l2\nu$: HIGG-2016-07}

\subsubsection*{\boldmath $H\to \tau\tau$: HIGG-2017-07}

\subsubsection*{\boldmath $H\to b\bar b$: HIGG-2016-29 (VH) and HIGG-2016-30 (VBF)}

HIGG-2016-29 provides in Fig.~5 the $1\sigma$ intervals for the $ZH$ and $WH$  production modes. 



\subsubsection*{\boldmath $ttH$}

\subsubsection*{\boldmath $H\to inv$: HIGG-2016-28}

Results from the search for invisibly decaying Higgs bosons produced in association with a $Z$ boson are presented in \cite{Aaboud:2017bja}. Assuming the Standard Model $ZH$ production cross-section, an observed (expected) upper limit of 67\% (39\%) at the 95\% confidence level is set on BR$(H\to inv)$ for $m_H= 125$~GeV. We use $1-{\rm CLs}$ as function BR$(H\to inv)$ extracted from auxiliary Figure~1c on the analysis' webpage. 



%-----------------------------------------------------------------------------------------------
\subsection{CMS Run~2 results for 36~fb$^{-1}$}
%-----------------------------------------------------------------------------------------------

The CMS Run~2 results included in this release are summarised in Table~\ref{tab:CMSresults} and explained in more detail below.

\begin{table}[h]\centering
\begin{tabular}{l | ccccccc}
mode & $\gamma\gamma$ & $ZZ$ & $WW$ & $\tau\tau$ & $b\bar b$ & $\mu\mu$ & inv. \\
\hline
ggH & \cite{Sirunyan:2018koj} & \cite{Sirunyan:2018koj} & \cite{Sirunyan:2018koj} & \cite{Sirunyan:2018koj} & \cite{Sirunyan:2018koj} & \cite{Sirunyan:2018koj} & \cite{Sirunyan:2018owy} \\
VBF &  \cite{Sirunyan:2018koj} & \cite{Sirunyan:2018koj} & \cite{Sirunyan:2018koj} & \cite{Sirunyan:2018koj} &-- & \cite{Sirunyan:2018koj} & \cite{Sirunyan:2018owy} \\
WH &  \cite{Sirunyan:2018koj} & \cite{Sirunyan:2018koj} & \cite{Sirunyan:2018koj} & \cite{Sirunyan:2018cpi} & \cite{Sirunyan:2018koj} & -- & \cite{Sirunyan:2018owy} \\
ZH & \cite{Sirunyan:2018koj} & \cite{Sirunyan:2018koj} & \cite{Sirunyan:2018koj} & \cite{Sirunyan:2018cpi} & \cite{Sirunyan:2018koj} & -- & \cite{Sirunyan:2018owy} \\
ttH & \cite{Sirunyan:2018koj} & \cite{Sirunyan:2018koj} & \cite{Sirunyan:2018koj} & \cite{Sirunyan:2018koj} & \cite{Sirunyan:2018koj} & -- & -- \\
\end{tabular}
\caption{Overview of CMS Run~2 results included in this release. Note that we use the full $24\times 24$ correlation matrix 
for the signal strengths for each production and decay mode combination provided in \cite{Sirunyan:2018koj}.}
\label{tab:CMSresults}
\end{table}



{\bf\boldmath Combined measurements (HIG-17-031):} 
CMS presented in \cite{Sirunyan:2018koj} a combination of the individual measurements for the 
$H\to \gamma\gamma$~\cite{Sirunyan:2018ouh}, $ZZ$~\cite{Sirunyan:2017exp}, $WW$~\cite{Sirunyan:2018egh}, 
$\tau\tau$~\cite{Sirunyan:2017khh}, $b\bar b$~\cite{Sirunyan:2017elk,Sirunyan:2017dgc} and $\mu\mu$~\cite{Sirunyan:2018hbu} 
decay modes as well as the $t\bar tH$ analyses~\cite{Sirunyan:2018shy,Sirunyan:2018mvw,Sirunyan:2018ygk}. 
We use the best fit values and uncertainties for the signal strengths for each production %(ggH, VBF, WH, ZH, ttH) 
and decay  %($\gamma\gamma$, $ZZ$, $WW$, $\tau\tau$, $b\bar b$, $\mu\mu$) 
mode combination presented in Table~3 of \cite{Sirunyan:2018koj} together with the $24\times 24$ (!) correlation matrix 
provided in `Additional Figure~1' on the analysis webpage. As shown in Fig.~\ref{fig:validation_cms_combination}, 
this allows to reproduce well the coupling fits of the CMS paper.\\

\begin{figure}[h!]\centering
\includegraphics[width=0.5\textwidth]{validation/CMS/HIG-17-031-CVCF.pdf}
%\hspace*{-12mm}\includegraphics[width=0.43\textwidth]{validation/CMS/HIG-17-031-CgCGa_BRinvBRund_profiled.pdf}%
%\hspace*{-12mm}\includegraphics[width=0.43\textwidth]{validation/CMS/HIG-17-031-CgCGa_BRinvBRund_profiled.pdf} 
\caption{Fit of $C_F$ vs.\ $C_V$ using best fit values and uncertainties for the signal strengths for each production (ggH, VBF, WH, ZH, ttH) 
and decay ($\gamma\gamma$, $ZZ$, $WW$, $\tau\tau$, $b\bar b$, $\mu\mu$) mode combination together with the 
$24\times 24$ correlation matrix from the CMS combination paper~\cite{Sirunyan:2018koj}.}
\label{fig:validation_cms_combination}
\end{figure}


{\bf\boldmath $VH$, $H\to\tau\tau$ (HIG-18-007)}: The above data from \cite{Sirunyan:2018koj} is supplemented by the results 
for the $\tau\tau$ decay mode from the $WH$ and $ZH$ targeted analysis \cite{Sirunyan:2018cpi}. These are implemented in the 
form of 1D intervals for $\mu(ZH,\;H\to\tau\tau)$ and $\mu(WH,\;H\to\tau\tau)$ taken from Fig.~6 of \cite{Sirunyan:2018cpi}. \\

{\bf\boldmath $H\to$~invisible (HIG-17-023)}: 
In \cite{Sirunyan:2018owy}, CMS performed a search for invisible decays of a Higgs boson produced through vector boson fusion. 
We use the profile likelihood ratios for the qqH-tag, Z(ll)H-, V(qq')H- and ggH-tag categories extracted 
from their Fig.~8b together with the relative contributions from the different Higgs production mechanisms  
given in Table~6 of that paper. This assumes that the relative signal contributions stay roughly the same as for 
SM production cross sections. For validation, we reproduce in Fig.~\ref{fig:validation_cms_inv}
 the $C_g$ vs.\ $C_\gamma$ fit of \cite{Sirunyan:2018koj}, where the branching ratios of invisible and undetected decays 
are treated as free parameters.\footnote{The profiling was done with Minuit. Since Minuit does not allow conditional limits, like 
${\rm BR}(H\to {\rm inv.})+{\rm BR}(H\to {\rm undetected})<1$, we demanded that both BR$(H\to {\rm inv.})$ and BR$(H\to {\rm undetected})$ 
stay below 50\%. This explains in part the small difference to the official CMS result.}

\begin{figure}[h!]\centering
\includegraphics[width=0.5\textwidth]{validation/CMS/HIG-17-031-CgCGa_BRinvBRund_profiled.pdf}
%\hspace*{-12mm}\includegraphics[width=0.43\textwidth]{validation/CMS/HIG-17-031-CgCGa_BRinvBRund_profiled.pdf}%
%\hspace*{-12mm}\includegraphics[width=0.43\textwidth]{validation/CMS/HIG-17-031-CgCGa_BRinvBRund_profiled.pdf} 
\caption{Fit of $C_g$ vs.\ $C_\gamma$ using the data from the combined CMS measurement~\cite{Sirunyan:2018koj} and the 
search for invisible decays of a Higgs boson~\cite{Sirunyan:2018owy}. The branching ratios of invisible and undetected decays 
are treated as free parameters in the fit.}
\label{fig:validation_cms_inv}
\end{figure}

