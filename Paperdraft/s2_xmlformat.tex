\clearpage
%===================================================================================
\section{Extended XML format for experimental input} \label{sec:xml}
%===================================================================================

In the {\tt Lilith} database, every single experimental result is stored in a separate XML file. 
This allows to easily select the results to use in a fit, and it also makes maintaining and updating the database rather easy. 

The root tag of each XML file is {\tt <expmu>}, which has two mandatory attributes, {\tt dim} and {\tt type} to specify the type of signal strength result. 
Production and decay modes are specified via {\tt prod} and {\tt decay} attributes either directly in the  {\tt <expmu>} tag or as efficiencies in 
{\tt <eff>} tags. Additional (optional) information can be provided in {\tt <experiment>}, {\tt <source>},  {\tt <sqrts>},  {\tt <CL>} and  {\tt <mass>} tags. 
Taking the $H\to\gamma\gamma$ result from the combined ATLAS and CMS Run~1 analysis~\cite{Khachatryan:2016vau} 
as a concrete example, the structure of the XML file is 
 
\begin{verbatim}
<expmu decay="gammagamma" dim="2" type="n">
  <experiment>ATLAS-CMS</experiment>
  <source type="publication">CMS-HIG-15-002; ATLAS-HIGG-2015-07</source>
  <sqrts>7+8</sqrts>
  <mass>125.09</mass>
  <CL>68%</CL> 
  
  <eff axis="x" prod="ggH">1.</eff>
  <eff axis="y" prod="VVH">1.</eff>
  
  <!-- (...) -->
</expmu>
\end{verbatim}

\noindent 
where {\tt <!-- (...) -->} is a placeholder for the actual likelihood information. 
For a detailed description, we refer to the original {\tt Lilith} manual~\cite{Bernon:2015hsa}. 
In the following, we assume that the reader is familiar with the basic syntax. 

So far, the likelihood information could be specified in one or two dimensions % ({\tt dim="1"} or {\tt dim="2"})  
in the form of~\cite{Bernon:2015hsa}:  
1D intervals given as best fit with $1\sigma$ error; 
2D likelihood contours described as best fit point and parameters a, b, c which parametrize the inverse of the covariance matrix; 
or full likelihood information as 1D or 2D grids of $-2\log L$. 
The first two options, 1D intervals and 2D likelihood contours, 
declared as {\tt type="n"} in the  {\tt <expmu>} tag, employ an ordinary Gaussian approximation; 
in the 1D case, asymmetric errors are accounted for by putting together two one-sided Gaussians with the same mean but different variances, 
while the 2D case assumes symmetric errors.   
This does does not always allow to describe the experimental data (i.e.\ the true likelihood) very well. 
Full 2D likelihood grids would be much better but are rarely available. %\footnote{We note that \cite{Boudjema:2013qla} strongly advocated the publication of full likelihood grids in 2 or more dimensions but unfortunately this wasn't followed up by the experimental collaborations.} 

In order to treat asymmetric uncertainties in a better way, we have extended the XML format and fitting procedure in {\tt Lilith} to  
Gaussian distributions of variable width (``variable Gaussian'') as well as generalized Poisson distributions. 
The declaration is {\tt type="vn"} for variable Gaussian or {\tt type="p"} for Poisson distribution in the {\tt <expmu>} tag. 
Both work for 1D and 2D data with the same syntax. 
Moreover, in order to make use of the $N$-dimensional ($N>2$) correlation matrices which both ATLAS 
and CMS have started to provide, we have added a new XML format for correlated signal strengths in more than two dimensions. 
This can be used with the ordinary or variable Gaussian approximation for the likelihood. 
In the following we give explicit examples for the different possibilities. 

\clearpage

\subsection*{1D likelihood parameterization}

For 1D data, the format remains the same as in \cite{Bernon:2015hsa}. 
For example, a signal strength $\mu(ZH,\, b\bar b)=1.12^{+0.50}_{-0.45}$ is implemented as

\begin{verbatim}
  <bestfit>1.12</bestfit>
  <param>
    <uncertainty side="left">-0.45</uncertainty>
    <uncertainty side="right">0.50</uncertainty>
  </param>
\end{verbatim}

\noindent
The {\tt <bestfit>} tag contains the best-fit value, while  
the {\tt <uncertainty>} tag contains the left (negative) and right (positive) $1\sigma$ errors.\footnote{The values in the {\tt <uncertainty>} tag can be given with or without a sign.} 
The choice of likelihood function is done by setting  {\tt type="n"} for ordinary, 2-sided Gaussian (as in {\tt Lilith-1.1}); 
{\tt type="vn"} for a variable Gaussian; or {\tt type="p"} for a Poisson distribution in the {\tt <expmu>} tag.  


\subsection*{2D likelihood parameterization}

For {\tt type="vn"} and {\tt type="p"}, signal strengths in 2D with a correlation are now described in an analogous way as 1D data. 
For example,  $\mu({\rm ggH}, WW)=1.10^{+0.21}_{-0.20}$ and $\mu({\rm VBF}, WW)=0.62^{+0.36}_{-0.35}$ with a correlation 
of $\rho=-0.08$ can be implemented as 

\begin{verbatim}
<expmu decay="WW" dim="2" type="vn">
 
  <eff axis="x" prod="ggH">1.0</eff>
  <eff axis="y" prod="VBF">1.0</eff>

  <bestfit>
    <x>1.10</x>
    <y>0.62</y>
  </bestfit>
 
  <param>
    <uncertainty axis="x" side="left">-0.20</uncertainty>
    <uncertainty axis="x" side="right">+0.21</uncertainty>
    <uncertainty axis="y" side="left">-0.35</uncertainty>
    <uncertainty axis="y" side="right">+0.36</uncertainty>
    <correlation>-0.08</correlation>
  </param>
</expmu>
\end{verbatim}

\noindent
Here, the {\tt <eff>} tag is used to declare the {\tt x} and {\tt y} axes. 
The {\tt <bestfit>} tag specifies the location of the best-fit point in the ({\tt x,y}) plane. 
The {\tt <uncertainty>} tags contain the left (negative) and right (positive) $1\sigma$ errors for the {\tt x} and {\tt y} axes, and finally 
the {\tt <correlation>} tag specifies the correlation between {\tt x} and {\tt y}. 
The choice of likelihood function is again done by setting {\tt type="vn"} or {\tt type="p"}  in the {\tt <expmu>} tag. 

To ensure backwards compatibility, {\tt type="n"} however still requires the tags {\tt <a>}, {\tt <b>}, {\tt <c>} to give the inverse of
the covariance matrix % $C^{-1}=\left(\begin{array}{cc} a & b\\ b& c \end{array}\right)$ 
instead of {\tt <uncertainty>} and  {\tt <correlation>}, see \cite{Bernon:2015hsa}.


\subsection*{Multi-dimensional data}

For correlated signal strengths in more than 2 dimensions, a new format is introduced. 
We here illustrate it by means of the CMS result \cite{Sirunyan:2018koj}, 
which has signal strengths for 24 production and decay mode combinations plus a $24\times 24$ correlation matrix. 
First, we set {\tt dim="24"} and label the various signal strengths as axes d1, d2, d3, \ldots~d24:\footnote{The {\tt <experiment>}, {\tt <source>},  {\tt <sqrts>}, etc.\ tags are omitted for brevity.} 

\begin{verbatim}
<expmu dim="24" type="vn">
  <eff axis="d1" prod="ggH" decay="gammagamma">1.0</eff>
  <eff axis="d2" prod="ggH" decay="ZZ">1.0</eff>
  <eff axis="d3" prod="ggH" decay="WW">1.0</eff>
  ...
  <eff axis="d24" prod="ttH" decay="tautau">1.0</eff>  
\end{verbatim}
The best-fit values for each axis are specified as 
\begin{verbatim}
  <bestfit>
    <d1>1.16</d1>
    <d2>1.22</d2>
    <d3>1.35</d3>
    ...
    <d24>0.23</d24>
  </bestfit>
\end{verbatim}
The {\tt <param>} tag then contains the uncertainties and correlations in the form
\begin{verbatim}
  <param>
    <uncertainty axis="d1" side="left">-0.18</uncertainty>
    <uncertainty axis="d1" side="right">+0.21</uncertainty>
    <uncertainty axis="d2" side="left">-0.21</uncertainty>
    <uncertainty axis="d2" side="right">+0.23</uncertainty>
    ...
    <uncertainty axis="d24" side="left">-0.88</uncertainty>
    <uncertainty axis="d24" side="right">+1.03</uncertainty>
	
    <correlation entry="d1d2">0.12</correlation>
    <correlation entry="d1d3">0.16</correlation>
    <correlation entry="d1d4">0.08</correlation>
    ...
    <correlation entry="d23d24">0</correlation>   
  </param>
</expmu>
\end{verbatim}

\noindent
This will also work for {\tt type="n"}.
