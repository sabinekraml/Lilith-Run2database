\clearpage
%===================================================================================
\section{Extended XML format} \label{sec:xml}
%===================================================================================

In the Lilith database, every single experimental result is stored in a different XML file. 
The original input formats from~\cite{Bernon:2015hsa} are 
\begin{itemize} 
\item 1D intervals: best fit with $1\sigma$ errors; 
\item 2D likelihood contours: best fit, confidence level and parameters a, b, c which parametrize the inverse of the covariance matrix;
\item full likelihood information as 1D or 2D grids of $-2\log L$.
\end{itemize}
For a detailed discussion and description of the XML format, we refer the reader to the original Lilith manual~\cite{Bernon:2015hsa}. 
Here we just note that the first two options, 1D intervals and 2D likelihood contours, rely on an ordinary Gaussian approximation, 
which does not always describe the experimental data (i.e.\ the true likelihood) very well. 
Full 2D likelihood grids would be ideal but are rarely available.\footnote{We note that \cite{Boudjema:2013qla} strongly advocated 
the publication of full likelihood grids in 2 or more dimensions but unfortunately this wasn't followed up by the experimental collaborations.} 
We have therefore extended the XML format and fitting procedure in Lilith to include 
\begin{itemize} 
\item Gaussian distributions of variable width (``variable Gaussian'') and 
\item generalized Poisson distributions. 
\end{itemize}
For 1D and 2D data, the format follows the structure defined in \cite{Bernon:2015hsa}. The only difference is setting {\tt type="vn"} for a variable Gaussian and
{\tt type="p"} for a Poisson distribution in the {\tt <expmu>} tag. % (with {\tt dim="1"} for 1D and {\tt dim="2"} for 2D as before). 
The computation of the likelihood in {\tt computelikelihood.py} then follows Section~3.6 ``Variable Gaussian (2)'' of \cite{Barlow:2004wg}. 
Poisson distribution according to Section~3.4 ``Generalised Poisson'', eq.~(10a), of \cite{Barlow:2004wg}. 

This also now works for 2D data with a correlation.  
Taking the ATLAS result for $\mu(VBF, ZZ)$ and $\mu(ggH, ZZ)$ with correlation $\rho=-0.41$ from HIGG-2016-22 as an example:

\begin{verbatim}
<expmu decay="ZZ" dim="2" type="vn">
  <experiment>ATLAS</experiment>
  <source type="published">HIGG-2016-22</source>
  <sqrts>13</sqrts>
  <mass>125.09</mass>
  <CL>68\%</CL>  <!-- optional -->

  <eff axis="x" prod="VBF">1.0</eff>
  <eff axis="y" prod="ggH">1.0</eff>

  <bestfit>
    <x>4.0</x>
    <y>1.11</y>
  </bestfit>
 
  <param>
    <uncertainty axis="x" side="left">-1.46</uncertainty>
    <uncertainty axis="x" side="right">+1.75</uncertainty>
    <uncertainty axis="y" side="left">-0.21</uncertainty>
    <uncertainty axis="y" side="right">+0.23</uncertainty>
    <correlation>-0.41</correlation>
  </param>
</expmu>
\end{verbatim}

\noindent
Here, the {\tt <bestfit>} tag specifies the location of the best-fit point in the ({\tt x,y}) plane. 
The {\tt <uncertainty>} tags contain the left (negative) and right (positive) $1\sigma$ errors for the {\tt x} and {\tt y} axes. 
The {\tt <correlation>} tag specifies the correlation between {\tt x} and {\tt y}. 
The computation of the likelihood again follows Section~3.6 ``Variable Gaussian (2)'' of \cite{Barlow:2004wg}. 
Same for Poisson distribution, just setting {\tt type="p"}.

\subsection{Covariance matrices for ${\rm dim}>2$}

